% contexto, prop, metodol.
\begin{resumo}
\lettrine[lines=3]{O}\ desenvolvimento orientado a modelos (\textit{Model Driven Development} - MDD) � uma abordagem que tem ganhado cada vez mais espa�o na ind�stria e na academia, trazendo grandes benef�cios, como o aumento de produtividade, por exemplo. Uma forma de se trabalhar usando MDD em equipe � usando uma IDE associada a um sistema de versionamento. Entretanto, trabalhar colaraborativamente usar uma IDE associada a um sistema de versionamento pode trazer algumas complica��es para o desenvolvimento como: conflitos de modelos, documenta��o descontinuada, dificuldades por parte dos interessados em usar sistemas de versionamento etc.

Nesse contexto, este trabalho prop�e uma abordagem de uso de wiki para desenvolvimento de MDD, de modo que o desenvolvedor seja capaz de criar modelos, gerar c�digo-fonte, compartilhar e versionar os modelos e ainda documentar colaborativamente, de maneira mais simples e f�cil do que abordagens tradicionais, possibilitando que mais usu�rios n�o desenvolvedores possam participar mais no processo de desenvolvimento e ainda possibilitando o aumento de produtividade. 

Para tentar evidenciar de que � poss�vel uma wiki ser usada para desenvolver software, foi criada uma \textit{Domain Specific Language} - DSL em uma wiki e foram realizados 3 estudos de caso: um com estudantes do ensino m�dio e que representam os n�o desenvolvedores, um com quatro alunos de p�s-gradua��o e com experi�ncia de desenvolvimento na ind�stria e o �ltimo estudo de caso foi realizado com quarenta e oito participantes entre desenvolvedores e alunos de p�s-gradua��o em Ci�ncias da Computa��o. Os estudos de caso mostraram que � vi�vel usar uma wiki para desenvolvimento, que n�o desenvolvedores se adaptam bem � abordagem e que a 85\% dos desenvolvedores usariam uma wiki para desenvolver usando MDD. Os estudos tamb�m levantaram as principais barreiras para aumentar a aceita��o da abordagem.

Com isso, este trabalho apresenta al�m de uma abordagem relativamente in�dita na literatura, resultados sobre uso de sistemas de versionamento, de IDEs e de colabora��o.

\end{resumo}

