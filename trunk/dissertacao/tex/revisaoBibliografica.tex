\newpage

\chapter{Revis�o Bibliogr�fica}
\label{revisaoBibliografica}

\section{Model Driven Development}
\label{mdd}
Defini��o ...

Pq MDD? Vantagens do uso de Modelos (ainda � poss�vel checar se um modelo � satisfat�rio ou nao, mas um c�digo nao � poss�vel) e do MDD, A importancia do MDD

Justificativa de MDD na abordagem: Modelos sao colaborativos e mais faceis de entender por nao desenvolvedores, 

Uma crian�a pode programar usando uma wiki mas nao usando eclipse e svn, mesmo que no eclipse seja modelos. A crianca � do dominio e o desenvolvedor o especialista// nivel da solu��o: java, c .... nivel do problema: UML, p ex //

1-C�digo-fonte � mais complicado de ser desenvolvido por mais de uma pessoa, logo � tamb�m menos reutiliz�vel. S� o especialista consegue fazer e nao o cara do dominio. -> falar mal do desenvolvimento tradicional // 2- Modelos s�o mais colaborativos, simples, gen�ricos e reus�veis (nao falar em reuso pra nao criar uma expectativa errada pro leitor) e permitem que n�o especialistas possam desenvolver/acompanhar; alem de poder-se checar por conformidade os modelos



\section{Domain Specific Languages}
\label{dsl}
Uma abordagem de MDD...

\subsection{Templates}
\label{templates}

\section{Colabora��o}
\label{colaboracao}
o que � colabora��o e quais seus benef�cios?

Como as pessoas colaboram no desenvolvimento de software, Falar de como � feito o desenvolvimento tradicional colaborativo -> Arquitetura do desenvolvimento tradicional 

\section{Wiki}
\label{wiki}
3- Artefatos online (wikis, gdocs) s�o mais colaborativos que controles de versao, bug tracks, ou outras ferramentas de gestao de configura��o por terem interfaces mais amig�veis -> Assim a Web aumenta ainda mais colabora��o: justificativa para wikis

Colaboracao e MDD: SVN

Colabora��o na wiki
