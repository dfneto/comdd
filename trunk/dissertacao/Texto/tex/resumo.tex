% contexto, prop, metodol.
\begin{resumo}
\lettrine[lines=3]{O}\ desenvolvimento orientado a modelos (MDD) � uma abordagem que tem ganhado cada vez mais espa�o na ind�stria e na academia, trazendo grandes benef�cios, como o aumento de produtividade, por exemplo. Contudo, ferramentas que auxiliem o desenvolvimento orientado a modelos de maneira colaborativa s�o praticamente inexistentes. Isso faz com que o MDD seja limitado ao desenvolvimento isolado e \textit{stand alone}, dificultando que sistemas complexos e abrangentes sejam desenvolvidos com esse paradigma. Assim, este trabalho apresenta uma abordagem colaborativa orientada a modelos (CoMDD), a qual pretende mostrar que a colabora��o pode trazer benef�cios ao MDD e permitir que desenvolvedores de MDD possam trabalhar colaborativamente, aumentando ainda mais a produtividade, possibilitando a troca de experi�ncias e conhecimentos, e permitindo o desenvolvimento de sistemas complexos. Ser�o realizados dois estudos de caso: um no dom�nio de tecnologia de informa��o (TI) e outro no dom�nio de sistemas embarcados (SE).
\end{resumo}
