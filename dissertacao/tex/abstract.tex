% contexto, prop, metodol.
\begin{abstract}
\lettrine[lines=3]{T}\ he Model Driven Development is an approach that gain more space in industry and academia, bringing great benefits such as increased productivity, for example. One way of working in teams with MDD is using an IDE with a versioning system. However, to work collaboratively with an IDE and a versioning system may have implications and problems for the development as: conflicts of models, documentation discontinued, difficulties for stakeholders to use versioning systems etc.

In this context, this paper proposes an approach to using the wiki to develop MDD, so that the developer is able to create models, generate source code, sharing and versioning models and also to document collaboratively, in a more simple and easy way than the traditional approaches. This enables non developers can participate  more in the development process and also allowing for increased productivity.

To try show that a wiki can be used to develop software, we created a Domain Specific Language (DSL) in a wiki and were performed three case studies: one with high school students and represent the non developers, another one with four posgraduate students and experience in software development in the industry, and the last case study was conducted with forty-eight participants between developers and postgraduate students in Computer Science. The case studies showed the feasibility of using a wiki for development, that non developers adapted well to the approach and the 85 \% of the developers would use a wiki to develop MDD. The study also raised the main barriers to increasing the acceptance of the approach.

Therefore, this paper presents also a relatively new approach in the literature and results on the use of versioning systems, IDEs and about collaboration.

%\textbf{Keywords:} scene segmentation, semantic segmentation, scene identification, scene boundary identification, multimodal technique.

\end{abstract}


